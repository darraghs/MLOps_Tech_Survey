
\documentclass[12pt, a4paper]{article}
\usepackage{graphicx}
\usepackage{blindtext}

% Page formatting
\usepackage{geometry}
 \geometry{
 a4paper,
 total={170mm,257mm},
 left=35mm,
 right=20mm,
 top=20mm,
 bottom=20mm,
 }
 \linespread{1.5}
 \raggedright

\usepackage[english]{babel}
\usepackage[utf8]{inputenc}
\usepackage{amsmath}
\usepackage{csquotes}% Recommended

\usepackage[style=authoryear-ibid,backend=biber]{biblatex}

\addbibresource{mlops_lit_review.bib}% Syntax for version >= 1.2
%\bibliography{./ref_biblatex} % the ref.bib file

 % Metda data
\title{A technological survey of Machine Learning Operations Platforms}
\author{Darragh Sherwin}
\date{February 2023}



\begin{document}
\maketitle
\begin{abstract}
In 2015, Google introduce the concept of hidden technical debt in machine learning systems\autocite{sculleyHiddenTechnicalDebt2015}. Machine learning systems offer the power to build complex useful prediction systems quickly. However, the maintenance of ML systems becomes difficult and expensive over time. To address, this hidden cost and complexity, a new set of practices called Machine Learning Operations (MLOps) sprung up. MLOps tries to take lessons from DevOps and apply them to ML. 
\linebreak
MLOps has been a significant area of growth in machine learning with Forbes estimating the market to be worth \$5 billion in 2025. However, there is little consensus on defining MLOps \autocite{mboweniSystematicReviewMachine2022}.
There is a need for a robust definition of MLOps and to understand how the current platforms address a robust definition of MLOps.

\end{abstract}

\section{Introduction}


\begin{enumerate}

\item Machine Learning systems are relatively fast and cheap to deploy but maintenance of production ML systems is difficult and expensive over time.  \autocite{sculleyHiddenTechnicalDebt2015}

\item The authors present several anti-patterns in machine learning engineering \autocite{muralidharUsingAntiPatternsAvoid2021}. These antipatterns are used to demonstrate MLOps mistakes. However, the authors fail to demonstrate how MLOps could be used to solve the issues arising from the antipatterns. A number of the presented antipatterns like \textit{Tuning-under-the-carpet} and \textit{Bad Credit Assignment} could be addressed by offline experimentation tracking, a core component of MLOps.

\item From 60 studies reviewed, there was no clear definition of MLOps but there were common elements. This has caused ambiguity and misconceptions. There is a need to understand how to implement MLOps across the industry \autocite{mboweniSystematicReviewMachine2022}. 

\item In their interviews with machine learning engineers in various industries, they come up with the three V's of MLOps - Velocity, Validation and Versioning \autocite{shankarOperationalizingMachineLearning2022}. 
    \begin{enumerate}
        \item Velocity, prototype and iterate quickly due to the experimental nature of machine learning.
        \item Validation, test the changes, prune bad ideas and proactively monitor.
        \item Versioning, storing and managing multiple versions of datasets, pipelines and models for querying and debugging.
    \end{enumerate}
The authors interviewed 19 machine-learning engineers from approximately eight distinct industries in the study. This is a minimal set of interviews to draw significant conclusions from most industries that could use machine learning in a production setting. The study presented only one industry that may have regulations on the use of machine learning. An extensive set of responses is required to understand better the requirements to put machine learning into production.

\item In a survey of 331 professionals, 40\% of respondents responded they worked on both models and infrastructure, highlighting a need for better infrastructure around machine learning in production settings \autocite{makinenWhoNeedsMLOps2021}. The survey looked at three main areas:
\begin{enumerate}
    \item Data discovery 
    \item Model training and serving
    \item Managing several models in production along with associated pipelines
\end{enumerate}


\end{enumerate}

\printbibliography

\end{document}